\documentclass{beamer}
\documentclass[draft]{beamer}
\documentclass[handout]{beamer}
\documentclass{article}
\usepackage[ngerman]{babel}
\usepackage[latin1]{inputenc}
% Vorgefertigte Themen ermöglichen ein ansprechendes Äußeres ohne viel Aufwand einfach das % löschen
%
%
%\usetheme{AnnArbor }
%\usetheme{ Antibes }
%\usetheme{ Bergen |Berkeley | Berlin | Boadilla |boxes | CambridgeUS | Copenhagen |Darmstadt | default | Dresden |Frankfurt | Goettingen |Hannover |Ilmenau | JuanLesPins | Luebeck |Madrid | Malmoe | Marburg |Montpellier | PaloAlto | Pittsburgh |Rochester | Singapore | Szeged |Warsaw}
%\usetheme{ Berkeley | Berlin | Boadilla |boxes | CambridgeUS | Copenhagen |Darmstadt | default | Dresden |Frankfurt | Goettingen |Hannover |Ilmenau | JuanLesPins | Luebeck |Madrid | Malmoe | Marburg |Montpellier | PaloAlto | Pittsburgh |Rochester | Singapore | Szeged |Warsaw}
%\usetheme{ Berlin | Boadilla |boxes | CambridgeUS | Copenhagen |Darmstadt | default | Dresden |Frankfurt | Goettingen |Hannover |Ilmenau | JuanLesPins | Luebeck |Madrid | Malmoe | Marburg |Montpellier | PaloAlto | Pittsburgh |Rochester | Singapore | Szeged |Warsaw}
%\usetheme{ Boadilla |boxes | CambridgeUS | Copenhagen |Darmstadt | default | Dresden |Frankfurt | Goettingen |Hannover |Ilmenau | JuanLesPins | Luebeck |Madrid | Malmoe | Marburg |Montpellier | PaloAlto | Pittsburgh |Rochester | Singapore | Szeged |Warsaw}
%\usetheme{ boxes | CambridgeUS | Copenhagen |Darmstadt | default | Dresden |Frankfurt | Goettingen |Hannover |Ilmenau | JuanLesPins | Luebeck |Madrid | Malmoe | Marburg |Montpellier | PaloAlto | Pittsburgh |Rochester | Singapore | Szeged |Warsaw}
%\usetheme{ CambridgeUS | Copenhagen |Darmstadt | default | Dresden |Frankfurt | Goettingen |Hannover |Ilmenau | JuanLesPins | Luebeck |Madrid | Malmoe | Marburg |Montpellier | PaloAlto | Pittsburgh |Rochester | Singapore | Szeged |Warsaw}
%\usetheme{ Copenhagen |Darmstadt | default | Dresden |Frankfurt | Goettingen |Hannover |Ilmenau | JuanLesPins | Luebeck |Madrid | Malmoe | Marburg |Montpellier | PaloAlto | Pittsburgh |Rochester | Singapore | Szeged |Warsaw}
%\usetheme{ Darmstadt | default | Dresden |Frankfurt | Goettingen |Hannover |Ilmenau | JuanLesPins | Luebeck |Madrid | Malmoe | Marburg |Montpellier | PaloAlto | Pittsburgh |Rochester | Singapore | Szeged |Warsaw}
%\usetheme{ Dresden |Frankfurt | Goettingen |Hannover |Ilmenau | JuanLesPins | Luebeck |Madrid | Malmoe | Marburg |Montpellier | PaloAlto | Pittsburgh |Rochester | Singapore | Szeged |Warsaw}
%\usetheme{Frankfurt | Goettingen |Hannover |Ilmenau | JuanLesPins | Luebeck |Madrid | Malmoe | Marburg |Montpellier | PaloAlto | Pittsburgh |Rochester | Singapore | Szeged |Warsaw}
%\usetheme{ Goettingen }
%\usetheme{ Hannover}
%\usetheme{ Ilmenau }
%\usetheme{ JuanLesPins }
%\usetheme{ Luebeck}
%\usetheme{ Madrid }
%\usetheme{ Malmoe}
%\usetheme{ Montpellier}
%\usetheme{ PaloAlto }
%\usetheme{ Pittsburgh }
%\usetheme{ Rochester }
%\usetheme{ Singapore }
%\usetheme{ Szeged }
\usetheme{ Warsaw}
% % % % % % % % % % % % % % % % % % % % % % % % % % % % % % % % % % % % % % % % % % % % % % % % % % % % % % % % % % % %%%
%
%
%
%Farbthemen bestimmen die Farben einer Folie.
%
%\usecolortheme{ albatross}
%\usecolortheme{ beaver }
%\usecolortheme{  beetle }
%\usecolortheme{ crane }
%\usecolortheme{ dolphin }
%\usecolortheme{ dove }
%\usecolortheme{ fly }
%\usecolortheme{ lily }
%\usecolortheme{ orchid }
%\usecolortheme{ rose}
\usecolortheme{ seagull }
%\usecolortheme{ seahorse }
%\usecolortheme{ sidebartab }
%\usecolortheme{ structure }
%\usecolortheme{ whale }
%\usecolortheme{  wolverine }
%%%%%%%%%%%%%%%%%%%%%%%%%%%%%%%%%%%%%%%%%%%%%%%%%%%%%%%%%%%%%%%%%%%%%%%%%%%%%%
%
%Die Themen für Schriftzeichen bestimmen die Schriftart.
%
%
%\usefonttheme{ serif }
\usefonttheme{ professionalfonts }
%\usefonttheme{ structurebold  }
%\usefonttheme{ structureitalicserif  }
%\usefonttheme{ structuresmallcapsserif }
%
%%%%%%%%%%%%%%%%%%%%%%%%%%%%%%%%%%%%%%%%%%%%%%%%%%%%%%%%%%%%%%%%%%%%%%%%%%%%%%
%
%
%
%\useinnertheme{ rounded }
%\useinnertheme{ circles }
%\useinnertheme{ inmargin }
%\useinnertheme{ rectangles }
%
%%%%%%%%%%%%%%%%%%%%%%%%%%%%%%%%%%%%%%%%%%%%%%%%%%%%%%%%%%%%%%%%%%%%%%%%%%%%%%
%
%Äußere Themen spezifizieren die Grenzen einer Folie und sagen ob und wo die inneren Elemente liegen.
%
%\useoutertheme{ infolines }
%\useoutertheme{ miniframes }
%\useoutertheme{ shadow  }
%\useoutertheme{ sidebar }
%\useoutertheme{ smoothbars }
\useoutertheme{ smoothtree }
%\useoutertheme{ split }
%%%%%%%%%%%%%%%%%%%%%%%%%%%%%%%%%%%%%%%%%%%%%%%%%%%%%%%%%%%%%%%%%%%%%%%%%%%%%%
%Um halbtransparente Overlays auf seinen Folien zu haben, reicht es folgenden Schalter zu setzen:
%
%\setbeamercovered{transparent}
%%%%%%%%%%%%%%%%%%%%%%%%%%%%%%%%%%%%%%%%%%%%%%%%%%%%%%%%%%%%%%%%%%%%%%%%%%%%%%
%Zum Abschalten der kleinen Navigationsleiste am unteren Rand reicht folgende Zeile aus:
%
%\beamertemplatenavigationsymbolsempty
%%%%%%%%%%%%%%%%%%%%%%%%%%%%%%%%%%%%%%%%%%%%%%%%%%%%%%%%%%%%%%%%%%%%%%%%%%%%%%
%
%Seitenzahlen in Fußzeile einfügen
%
\setbeamertemplate{footline}[frame number]
%%%%%%%%%%%%%%%%%%%%%%%%%%%%%%%%%%%%%%%%%%%%%%%%%%%%%%%%%%%%%%%%%%%%%%%%%%%%%%
%
%Um die Metainfomationen zu setzen die unter anderem für die Titelseite verwendet werden, kann man sich folgender Befehle bedienen
%
%
\title[Kurzform]{Vortrag zur Berechenbarkeit}
%    Titel des Vortrages
\subtitle[Kurzform]{Untertitel}
%
\title[Kurzform]{Vortrag zur Berechenbarkeit}
%    Titel des Vortrages
\subtitle[Kurzform]{Untertitel}
%    Untertitel
\author[M. Schulz]{Michael Schulz}
%    Autor festlegen
\institute[IfI -- HU Berlin]{Institut für Informatik\\ Humboldt-Universität zu Berlin}
%    Angabe des Institutes
\date[26.05.06]{26. Mai 2006}
%    Datum der Präsentation, alternativ kann mittels \date{\today} auch das aktuelle Datum eingetragen werden.
\logo{\pgfimage[width=2cm,height=2cm]{hulogo}}
%    Die Datei hulogo.pdf (bzw. hulogo.png, hulogo.jpg, hulogo.mps bei Verwendung von pdftex als Backend) als Logo auf allen Folien, hier mithilfe des Paketes pgf.
\titlegraphic{\includegraphics[width=2cm,height=2cm]{hulogo}}
%    Die Datei hulogo.pdf (bzw. analog wie bei \logo auch entsprechendes Format) als Logo nur auf der Titelseite unter Verwendung des Paketes graphicx.
\subject{Berechenbarkeit}
%    Setzt Thema in den PDF-Dokument-Eigenschaften
\keywords{Gödelsätze, Unentscheidbarkeit}
%    Setzt Schlüsselwörter in den PDF-Dokument-Eigenschaften     
%%%%%%%%%%%%%%%%%%%%%%%%%%%%%%%%%%%%%%%%%%%%%%%%%%%%%%%%%%%%%%%%%%%%%%%%%%%%%%
%Die erste Folie
%
%Einzelne Folien erzeugt man mittels Frame-Umgebung. Ein Frame kann wiederum mehrere Slides enthalten, die etwa durch Seitenumbrüche oder Overlay-Aktionen entstehen.
\begin{frame}[Overlay-Aktionen][Optionen]{Titel}{Untertitel}
	Inhalt
\end{frame}

% Quelle: http://www2.informatik.hu-berlin.de/~mischulz/beamer.html
